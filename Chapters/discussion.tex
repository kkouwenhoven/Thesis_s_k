\chapter{Discussion}

As briefly discussed in chapter~\ref{chap:results} the accuracy of the results is low.
Since, at the time of writing, the prototype is not yet fully integrated measurements had to be done using an oscilloscope.
Using the oscilloscope in this set-up the resolution of the TDOA and delay measurements is \SI{0.1}{\milli\second}.
This corresponds to a distance resolution of \SI{3.4}{\centi\meter}.
In the final prototype these measurements will be done using a timer with a oscillator frequency of \SI{40}{\kilo\hertz} resulting in a resolution of \SI{25}{\micro\second}.
This resolution in time measurement gives a distance resolution of \SI{8}{\milli\meter}.
%By comparing these resolutions it's clear that, using the current results described in chapter~\ref{chap:results}, nothing can be said about the accuracy of the final fully integrated system.
%The results do show that even with the lower measurement resolution of the oscilloscope,
the required measurement resolution of \SI{10}{\centi\meter} from requirement~\ref{req:locres} is reached.

If we compare the resolution of \SI{8}{\milli\meter} to the error created by the uncertainty of the delay, \SI{17}{\centi\meter}, it's is clear that the final accuracy will be mainly determined by the uncertainty in the delay.
As discussed in chapter~\ref{chap:results} this error has a maximum of \SI{17}{\centi\meter} due to the Tx and Rx delay of the radio frequency pulse. Some imperfections of the antenna reflector result in an accuracy of \SI{20}{\centi\meter}.
This error is larger than the accuracy requirement \ref{req:locper} of \SI{10}{\centi\meter}.
This means that, without a filtering method the accuracy requirement is not met.

If we are able to characterise the distribution of the uncertainty in the delay the particle filter implemented on the MCU will be able to increase the accuracy.
How this particle filter works is discussed in~\cite{processing}.

In conclusion the results given in chapter~\ref{chap:results} do not give enough information to determine the final accuracy of the prototype system.
We can however make conclusions on which factors will have the largest influence on the accuracy.
The results do give enough data to conclude that the final system will be able to perform distance measurements with an accuracy close to the required performance.

The biggest limitation at this point in the project is the missing integration with the microcontroller.
If the microcontroller was integrated it would be possible to do a large set of measurement with a high resolution making it possible to draw conclusions on the accuracy of the system.
With the microcontroller integrated it is also possible to measure if the integration introduces an error.
As we've already seen the communication system had an unexpected large delay.
It could be possible that the integration with the microcontroller causes other delays.
At this point in time it is thus not possible to find all the delays in the complete system.

With the microcontroller present it is also possible to do better measurements on the communication channel delay.
It is important to characterize the distribution of the delay present in the communication channel to determine the average error introduced by the delay.
Currently only the max error is known (measured with a low resolution) but the average error can not yet be calculated.

If we want to fully test the prototype and draw meaningful conclusion about the accuracy of the prototype the full integration must be completed.
This full integration will be done in the final 2 weeks of the project which unfortunately means that they cannot be included in this thesis.
The steps required to complete this integration will be discussed in chapter~\ref{chap:recom}.
