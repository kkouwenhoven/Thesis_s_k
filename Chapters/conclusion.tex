\chapter{Conclusion}
\label{chap:conc}
The first decision to be made was to, based on the state of the art research, decide on a concept.
The chosen concept is based on the Cricket project, a research project from MIT \cite{Priyantha2000,Priyantha2005,Balakrishnan2003,Smith2005}, and uses the following technique:
to measure the distance between Zebros a RF message and a ultrasonic pulse are transmitted concurrently, the distance between Zebros can than be calculated from the TDOA.
With this technique we reach the key performance indicators (KPI) as given in the programme of requirements:

\begin{itemize}
  \item
    \ref{req:neighbours} \textbf{The module should be able to locate multiple neighbouring Zebros (KPI).}
    As described in chapter~\ref{chap:concept} both the ultrasonic pulse and RF pulse reach all Zebros within the range of the ultrasonic pulse.
    The range of the ultrasonic pulse is larger than the required distance sensing range of \SI{5}{\meter} \ref{req:locrange}.
    This means that all Zebros within the required distance sensing range receive the necessary information to perform a distance measurement.
    As long as all Zebros get a turn to transmit their ultrasonic pulse and RF message all Zebros in the swarm can be located.
  \item
    \ref{req:allunits} \textbf{The module should be able to be used on and locate charging stations and similar units (KPI).}
    The chosen technique does not require modules to be moving in order to perform measurements.
    The module can thus be implemented on static objects such as a charging station.
  \item
    \ref{req:distinguish} \textbf{The module should be able to distinguish different users (Zebro, charging station, other member of Zebro familie) of the module (KPI)}
    The RF message, which is transmitted concurrently with the ultrasonic pulse, contains the Zebro serial number of the transmitting module.
    Using this serial number other Zebros know who is transmitting: it's tells them what the transmitter is (a Zebro, a charging station etc.) and who the transmitter is (Zebro 50 or Zebro 74).
    Thus the module can distinguish between different users.
\end{itemize}

After the conceptual design stage, we made some proof of concept circuits and 3D printed the antenna for the ultrasonic transceiver.
During tests we encountered some problems with the receiving circuit: the circuit had noise oscillations which were amplified and resulted in undesired outputs.
The problem was solved with the use of op-amp with a higher bandwidth.
Other problems occurred at the peak detection circuit, the capacitor discharged to fast resulting in undesired outputs of the comparator stages.
This problem was solved by changing the values of the components in the peak detection circuit.
With all circuit problems solved we were able to proof the correct working of the transmitter and receiver concepts.
The task after the proof of concept was to combine these transmitter and receiver circuit to a working prototype.

To make the working prototype, the module needed to switch between transmitting and receiving mode to protect the receiver part form the 30 $V_{p-p}$ of the transmitter.
Several options were available, but we decided to use an analogue switch, also called a bilateral switch, for this problem.
With all components known and proofed on, a naked prototype PCB was designed.
This prototype, made with a PCB milling machine, was used for integration and testing with the communication module.
A problem we encountered was that the module we used for sending and receiving the RF signal has a delay between transmission and receiving of a package.
We need to take this delay into account for determining the distance.

During the testing phase we found the results, with the delay taken into account, correspond to the actual distance between modules.
Since we use the average value of the delay we get an error corresponding to the deviation in the delay.
Using the delay gives a max. error of $\pm$ \SI{17}{\centi\meter}.
This means we do not comply to requirement \ref{req:locper} which states that the accuracy has to be at least \SI{10}{\centi\meter}

During these tests the processing method was not yet implemented.
Using the processing method, a particle filter, we might be able to increase the accuracy of the module.
As long as the distribution of the delay can be characterized, the particle filter will be able to improve the accuracy of the system.
Tests with the implemented particle filter will tell if the accuracy of the full system complies to requirement \ref{req:locper}.
These test will be performed after the deadline of the thesis.

The resolution of the measurements is \SI{8}{\milli\meter} since the TDOA is measured using a timer with a frequency of \SI{40}{\kilo\hertz}.
Requirement \ref{req:locres} states that the required resolution is \SI{10}{\centi\meter}.
This requirement is thus easily reached by the designed system.

We thus conclude that there is a high possibility that the fully integrated system will reach the requirements set by the Zebro team.
Before we can test the performance of the module work need to be done on the final integration of all systems.
We give a recommendation on the task required to finish the final integration in the next chapter.
These tasks will be undertaken by our project group in the last two weeks between the thesis deadline and the thesis defence.

\section{Conclusion after integration}

In chapter~\ref{chap:results} the results for the system after integration are given.
From these results we conclude that the integrated system (still without processing) is capable of performing distance measurements.
All the conclusions drawn above still hold for the integrated system, the accuracy is still the same.
