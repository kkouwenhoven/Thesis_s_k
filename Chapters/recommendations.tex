\chapter{Recommendations}
\label{chap:recom}
Two types of recommendations are given:

The first type focusses on tasks necessary to integrate the full system and perform measurements.
These tasks are planned for the next two weeks and need to be completed before the module can be delivered to the Zebro team.
These recommendation are thus given as a list of tasks.

The second type of recommendations focusses on the future of the project.
How will the module fit into the Zebro project and how might the module change in the future of the Zebro project?
Possible improvements for the system will be discussed and topics which require more research are given.

\section{Tasks planned}

\begin{reqs}{R}
  \item
  Focus on integration of distance sensing systems with communication and processing system.
    \begin{subreqs}
      \item
      Connect distance sensing system to the microcontroller as described in section~\ref{sec:sys_int}.
      \item
      Implement counter measure for the delay in the communication channel as described in section~\ref{sec:delay}.
      \item
      Test if the microcontroller can calculate the TDOA of the ultrasonic pulse and the RF message.
    \end{subreqs}
  \item
  Focus on acquiring measurement results for full system.
    \begin{subreqs}
      \item
      Make testbench software which performs a large set of measurement for a single distance and stores the acquired results.
      \item
      Run tests for multiple distances.
      \item
      Analyse test results to determine accuracy and stability of fully integrated system.
    \end{subreqs}
  \item
  Focus on characterizing the distribution of the delay in the communication channel.
  \item
  Integrate module with Zebro power management system.
\end{reqs}

\section{Recommendations for the future of the module in the Zebro project}

With the tasks described above completed the module is ready to be implemented on the Zebro.
To check if the module works together small test have to be performed.
The most important of these test is to check if the module works on a waking Zebro.
It is possible that the movement of the leg influences the module performance.
If this is the case countermeasures can be taken.
The system can be calibrated or mechanically decoupled from the Zebro by using rubber dampeners in the mechanical connection.

It's possible that the reflector cone will become unstable when the Zebros moves at high speed or trough rough terrain.
If this happens it's possible to improve the stability by using 3 riser legs instead of 2.
During testing the effect of these risers on the transmitting and receiving behaviour was found to be negligible so using three risers will not affect the performance of the system.

More research could be done to improve the reflector antenna.
During the conceptual design the effect of material or the not completely smooth surface of the 3D print was not discussed.
It might be possible that a smooth metal reflector improves the transmitting and receiving behaviour antenna.

\subsection*{Zebro to the moon}

There are rumours of a Zebro going to the moon.
Our module does not work in the environment since we need the propagation of sound to do measurements.
If the Lunar Zebro requires a localization module we advise that research into an alternative to this module is performed as soon as possible.
During our state of the art research and conceptual design we found that most alternatives to the ultrasound system give poor performance.
RSSI for example has no direction relation to the distance between modules and is very dependent on device orientation.
Swarm networks further decrease the use of RSSI since the message are hopped in the network making it hard to find the RSSI value corresponding to a certain Zebro.
We believe that finding a working alternative to the system described in this paper requires time so we advice the Zebro team to look into alternative for the Lunar Zebro as soon as possible.
We estimate that the accuracy of these alternatives will be lower which has to be taken into account by the Zebro swarm behaviour group.
