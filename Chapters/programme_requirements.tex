\chapter{Programme of requirements}
\label{chap:por}

The current Zebro project has developed a few working Six-legged robots (Zebros).
These proof-of-concepts have shown that legged robots are possible and have definite advantages, especially on rough terrain.
However, these robots sometimes exhibit strange and undesired behavior.
This is currently attributed to the fact that all control algorithms run on a single central processor.

Furthermore, due to the high cost and effort to produce these robots, only a limited number of these robots have been built.
Moreover, every build robot is slightly different from the others.

The main purpose of the Zebro is swarm-related behaviour, where a swarm of relatively inexpensive, simple and straight-forward robots is able to perform tasks that are not suited for one bigger robot.
An example would be the mapping of a disaster area.

Numerous simulations of swarms are available, but realistic swarm behaviour needs to be studied using a realistically sized swarm.
Therefore, a larger number of Zebros is needed. Thus, it is highly desirable to redesign the main components for simplicity, producibility, repairability, stability, and cost.

The Zebro Team is currently working on a modularized Deci Zebro, the size of an A4 sheet of paper.
The Deci’s will form a swarm of a 100 robots by the end of June.

Currently the Deci’s have no means of determining the distance between themselves.
This capability allows the Deci to become a real swarming robot.
Besides giving information to the central ‘brain’ of the Deci, the module itself should also indicate to a human what the module ‘sees’ and which direction it wants to go.

\textbf{The goal of the bachelor project is the development of a new Localisation module for Zebro as well as a communication system for this module.}

The Localisation module will be a part of Zebro itself, and determine the Zebro’s position in the swarm and relative to e.g. the charging station.

\section{Functional Requirements}
\begin{reqs}{F}
  \item\label{req:neighbours} The module should be able to locate multiple neighbouring Zebros (KPI)
    \begin{subreqs}
      \item\label{req:allunits} The module should be able to be used on and locate charging stations and similar units (KPI)
      \item\label{req:distinguish} The module should be able to distinguish different users (Zebro, charging station, other member of Zebro familie) of the module (KPI)
    \end{subreqs}

  \item The module should be able to receive and transmit packets to neighbouring Zebros
    \begin{subreqs}
      \item The module should be able to make contact with at least 3 neighbouring Zebros
      \item The module should be able to broadcast housekeeping data for inspection by users
    \end{subreqs}
\end{reqs}


\section{System Requirements}
\begin{reqs}{S}
  \item\label{req:specs} The mechanical, power, and digital connections must conform to the specifications outlined in~\cite{DeciSpecs}

  \item The module must conform to the Zebrobus protocol outlined in~\cite{Zebrobus}

  \item The module must respect the modularity of Zebro modules
    \begin{subreqs}
    \item The module should be a slave to the Zebro bus and cannot make requests itself
      \item\label{req:inhibition} Use of the module cannot inhibit the functionality of other modules on the Zebro
      \item With the exception of data broadcast on the Zebro bus, the module should collect data exclusively using its own sensors
      \item The module should communicate its data to the Zebro by responding to requests over the Zebro bus
      \item The modules should be interchangeable between robots and easily replaceable
    \end{subreqs}

  \item The module should be able to be used in indoor environments

  \item The communications network should be homogeneous
    \begin{subreqs}
      \item Communications cannot rely on a predetermined master node or network hub
      \item An individual Zebro within range should be able to join the communications network at any time
      \item Any individual Zebro in the network should be able to leave without affecting network functionality
    \end{subreqs}

  \item\label{req:stationary} The module should be usable on stationary elements, such as a charging station, without any changes to the hardware

  \item The module connects to the Zebro power system (see~\ref{req:specs})
    \begin{subreqs}
      \item\label{req:volt}The power system is rated at \SIrange{13}{19}{\volt}
      \footnote{There are no specifications yet for the maximum drawn power/current}
    \end{subreqs}

  \item\label{req:debug} The module should be able to provide debugging information over the Zebro bus
\end{reqs}

\section{Performance Requirements}
\begin{reqs}{P}
  \item\label{req:locper} The module should be able to locate other Zebros with an accuracy of at least \SI{10}{\centi\meter}
  \item\label{req:locrange} The module should be able to locate Zebros within a range of at least \SI{5}{\meter}
  \item\label{req:locres} The distance data provided to the Zebro brain should have a resolution of \SI{10}{\centi\meter}
  \item\label{req:omnidir} The module should be able to locate Zebros in all directions of the horizontal plane (\SI{360}{\degree} coverage)
\end{reqs}

\section{Development Constraints}
\begin{reqs}{D}
  \item The module production cost shall not exceed \euro $150$ per unit
  \item The module PCB should support pick-and-place service
\end{reqs}
